\renewcommand{\arraystretch}{2.0}
\section{Tensorial and index notations}
In a fixed reference frame $ \left( \vect{e}_1 , \, \vect{e}_2 , \, \vect{e}_3  \right)$, taking the Einstein convention on indices into account:
%
\begin{description}
\item[$0^{th}$ order tensor: ] scalar $T$
\item[$1^{st}$ order tensor: ] vector $\vect{u} = u_i \, \vect{e}_i  $
\item[$2^{nd}$ order tensor: ] matrix $\tens{ \sigma} =  \sigma_{ij}  \,\vect{e}_i \otimes \vect{e}_j$
\item[$3^{rd}$ order tensor: ] $\vect{\tens{a}} = a_{ijk} \, \vect{e}_i \otimes \vect{e}_j  \otimes \vect{e}_k$
\item[\textcolor{white}{$3^{rd}$ order } $ \vdots $]
\item[$ n^{th} $ order tensor: ]$a^{(n)} = a_{i_1 i_2 \cdots i_n} \, \vect{e}_{i_1} \otimes \vect{e}_{i_2} \otimes \cdots  \otimes \vect{e}_{i_n}$
\end{description}

\begin{table}[!hptb]
\centering
\begin{tabular}{l| c| r@{ \,}c@{\,}l  @{\,}c@{\,}l @{\,}l }
Operators & Symbols & Formulae \\
\hline
tensorial product & $\otimes$ & $\vect{u} $ & $ \otimes $ & $\vect{u}$ & $= $& $u_i u_j $&  $ \vect{e}_i \otimes \vect{e}_j $ \\
                         &                  & $a^{(n)}$ & $ \otimes $ & $b^{(m)} $& $= $&  $a_{i_1  \cdots i_n} b_{j_1  \cdots j_m}$ & $\vect{e}_{i_1} \otimes \cdots \otimes \vect{e}_{i_n} \otimes \vect{e}_{j_1} \otimes \cdots \otimes \vect{e}_{j_m} $ \\
\hline
dot product&$.$ & $\vect{u} $&$ . $&$\vect{u} $&$ = $&$ u_i u_i  $ \\
                          & &$ a^{(n)} $&$ . $&$ b^{(m)} $&$ = $&$ a_{i_1  \cdots i_{n-1}k} b_{k j_2  \cdots j_m} $&$ \vect{e}_{i_1} \otimes \cdots \otimes \vect{e}_{i_{n-1}} \otimes \vect{e}_{j_2} \otimes \cdots \otimes \vect{e}_{j_m} $ \\
\hline
double dot &$ : $&$ \tens{\sigma} $&$ : $&$ \tens{\sigma} $&$ = $&$ \sigma_{ij} \sigma_{ij}    $ \\
       product& &$ a^{(n)} $&$ : $&$ b^{(m)} $&$ = $&$ a_{i_1  \cdots i_{n-2}kl} b_{k l j_3  \cdots j_m} $&$ \vect{e}_{i_1} \otimes \cdots \otimes \vect{e}_{i_{n-2}} \otimes \vect{e}_{j_3} \otimes \cdots \otimes \vect{e}_{j_m} $ \\
\hline
vectorial product &
$
\begin{array}{c}
 \times \\
\textrm{ \emph{or} } \\
\wedge
\end{array}
$
&$ \vect{u} $&$ \times $&$ \vect{v} $&$ = $&
$ \left( \begin{array}{c}
u_1\\
u_2 \\
u_3
\end{array} \right) \times
\left( \begin{array}{c}
v_1\\
v_2 \\
v_3
\end{array} \right)  $
&$ =
\left( \begin{array}{c}
u_2 v_3 - u_3 v_2\\
u_3 v_1 - u_1 v_3 \\
u_1 v_2 - u_2 v_1
\end{array} \right)
 $
\end{tabular}
\caption{Tensorial operators.}
\end{table}

%%%%%%%%%%%%%%%%%%%%%%%%%%%%%%%%%%%%%%%%%%%%%%%%%%
\section{Differential operators and standard relationships}
\begin{equation*}
\begin{array}{c}
%
\begin{array}{r c l @{\hspace{3cm}} r c l }
\dive \left(  \rot \, \vect{u}\right) &=&0  &
\rot  \left(  \gradv \, T\right)  &=&0 \\
\dive \left( \alpha  \vect{u} \right) &=& \alpha \dive \, \vect{u} + \gradv \, \alpha . \vect{u}  &
\rot \left( \alpha  \vect{u} \right) &=& \alpha \rot \, \vect{u} + \gradv \, \alpha  \times  \vect{u} \\
\dive \left(   \vect{u} \times \vect{v} \right) &=& \vect{v}. \rot \, \vect{u} - \vect{u} . \rot \, \vect{v} &
\rot \left( \rot \, \vect{u} \right) & = & \grad \left( \dive \, \vect{u} \right)  - \vect{\Delta } \, \vect{u}
\end{array}
\\
\grad \left( \vect{u}. \vect{v} \right) = \left( \ggrad  \vect{u} \right) . \vect{v}  \, + \,  \left( \ggrad  \vect{v} \right) . \vect{u} \, + \, \vect{u} \times \rot \, \vect{v}
\, + \,  \vect{v} \times \rot \, \vect{u} \\
\end{array}
\end{equation*}
%
%
%
\newpage
\begin{table}[!htb]
\centering
\begin{tabular}{l | c | l | l @{}l@{\,} l  }
Operators & Symbols & Definitions & Cartesian formulae \\
\hline
gradient & $\nabla $ & $\nabla^{(n+1)} \left[ a^{(n)} \right] \, . \, \vect{ \mathrm{d}x} = \left. \mathrm{d} a^{(n)} \right|_{ \vect{ \mathrm{d}x}}$  & $\vect{\nabla} \,T = \dfrac{\partial T}{ \partial x_i} \vect{e}_i$ \\
&\emph{or}&& $\tens{\nabla} \, \vect{u} = \dfrac{\partial \vect{u}}{ \partial x_j} \otimes \vect{e}_j $&$ = $&$ \dfrac{\partial u_i }{ \partial x_j}  \vect{e}_i \otimes \vect{e}_j $ \\
& $\grad$ && $\nabla^{(n+1)} \left[ a^{(n)} \right] $&$ = $&$ \dfrac{\partial a^{(n)} }{ \partial x_{i_{n+1}} } \otimes \vect{e}_{i_{n+1}} $ \\
\hline
divergence & $\nabla .$ & $\nabla^{(n-1)}. \left[ a^{(n)} \right] = \nabla^{(n+1)} \left[ a^{(n)} \right] : \tens{1} $  &
$ \divv \, \vect{u} = \dfrac{\partial u_i}{ \partial x_i}$ \\
& \emph{or} && $ \divt \, \tens{\sigma} = \dfrac{\partial  \sigma_{ij}}{ \partial x_j} \vect{e}_i$ \\
&  $\divv$  && $ \divv^{(n-1)} \left[ a^{(n)} \right] $&$ = $&$ \dfrac{\partial   a_{i_1  \cdots i_n} }{ \partial x_{i_n}} \vect{e}_{i_1} \otimes \cdots \otimes \vect{e}_{i_{n-1}} $ \\
\hline
Laplacian & $\Delta$  & $ \Delta a^{(n)} = \divv^{(n)} \left\{ \nabla^{(n+1)} \left[ a^{(n)} \right]  \right\}$ & $ \Delta  \,T = \dfrac{ \partial^2 T}{\partial x_i \partial x_i}$ \\
&\emph{or}  && $ \vect{ \Delta} \, \vect{u} =  \dfrac{ \partial^2 \vect{u}}{\partial x_j \partial x_j} $&$ = $&$ \dfrac{ \partial^2 u_i}{\partial x_j \partial x_j} \vect{e}_i$ \\
& $\nabla^2$ && $ \Delta^{(n)} a^{(n)}  $&$ = $&$ \dfrac{ \partial^2  a^{(n)}}{\partial x_{i_{n+1}} \partial x_{i_{n+1}}} $\\
&&&&$ = $&$ \dfrac{ \partial^2 a_{i_1  \cdots i_n}}{ \partial x_{i_{n+1}} \partial x_{i_{n+1}}} \vect{e}_{i_1} \otimes \cdots \otimes \vect{e}_{i_{n}} $ \\
\hline
rotational &
$
\begin{array}{c}
\vect{\nabla} \times\\
\textrm{ \emph{or} } \\
\rot
\end{array} $
&  & $\rot \, \vect{u}  =
 \left( \begin{array}{c}
\frac{\partial }{\partial x_1}\\
\frac{\partial }{\partial x_2}\\
\frac{\partial }{\partial x_3}
\end{array} \right)
$&$ \times $&$
\left( \begin{array}{c}
u_1\\
u_2 \\
u_3
\end{array} \right)
 =
\left( \begin{array}{c}
\frac{\partial u_3}{\partial x_2} - \frac{\partial u_2}{\partial x_3}\\
\frac{\partial u_1}{\partial x_3} - \frac{\partial u_3}{\partial x_1} \\
\frac{\partial u_2}{\partial x_1} -\frac{\partial u_1}{\partial x_2}
\end{array} \right)
$
\end{tabular}
\caption{Differential operators.}
\end{table}
%
%

\paragraph{Stokes Theorem}
%
\begin{equation*}
\int_{ \mathcal{S}} \rot \, \vect{u} \, . \, \vect{ \mathrm{d}S} = \int_{\partial \mathcal{S} } \vect{u} \, . \, \vect{\mathrm{d}l }
\end{equation*}
where $ \vect{ \mathrm{d}S }$ is the outward surface element.

\paragraph{Divergence theorem [Green-Ostrogradski]}
%
\begin{equation*}
\int_{ \Omega } \divv^{(n-1)}  \left[ a^{(n)} \right] \, \mathrm{d}\Omega= \int_{\partial \Omega } a^{(n)} \, . \, \vect{ \mathrm{d}S }
\end{equation*}
\paragraph{Rotational theorem}
%
\begin{equation*}
\int_{ \Omega } \rot \, \vect{u} \, \mathrm{d}\Omega= \int_{\partial \Omega } \vect{u} \, \times \, \vect{ \mathrm{d}S }
\end{equation*}

\paragraph{Leibnitz theorem}
%
\begin{equation*}
\dfrac{ \mathrm{d} }{ \mathrm{d}t }\int_{ \Omega } a^{(n)} \, \mathrm{d}\Omega =
\int_{ \Omega } \dfrac{\partial a^{(n)} }{ \partial t} \, \mathrm{d}\Omega
+ \int_{\partial \Omega } a^{(n)} \vect{v} \, . \, \vect{ \mathrm{d}S }
\end{equation*}
where $\vect{v}$ is control volume $ \Omega$ velocity.
